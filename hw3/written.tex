	\documentclass[11pt]{article}
	
	\title{Homework 1 - DM CS6220}
	\author{Nakul Camasamudram}
	\usepackage{amsmath,amsfonts,amsthm} % Math packages
	\usepackage{mathtools}
	\usepackage{adjustbox}
	\usepackage{soul}
	%----------------------------------------------------------------------------------------
	%	TITLE SECTION
	%----------------------------------------------------------------------------------------
	
	\newcommand{\horrule}[1]{\rule{\linewidth}{#1}} 
	
	\title{	
	\normalfont \normalsize 
	\textsc{Northeastern University, Data Mining Techniques - CS6220 Fall 2017} \\
	 % Your university, school and/or department name(s)
	\horrule{0.5pt} \\[0.4cm] % Thin top horizontal rule
	\huge Solutions to Homework 3, Part 1 \\ % The assignment title
	\horrule{2pt} \\[0.5cm] % Thick bottom horizontal rule
	}
	\author{Nakul Camasamudram} % Your name
	\date{\normalsize\today} % Today's date or a custom date
	\begin{document}
	
	\maketitle % Print the title
	\newpage

\section*{1. PCA Eigenvector Orthogonality}

\textbf{Solution:}\\

\noindent
Given,
\begin{align*}
	A\vec{x} &= \lambda_{1}\vec{x}\\
	A\vec{y} &= \lambda_{2}\vec{y}\\
\end{align*}

Multiply each equation above with the transpose of the other eigenvector

\begin{align*}
	\vec{y}^{\, t}A\vec{x} &= \lambda_{1}\vec{y}^{\, t}\vec{x}\\
	\implies \vec{x}^{\, t}A\vec{y} &= \lambda_{1}\vec{x}^{\, t}\vec{y}\\
	\vec{x}^{\, t}A\vec{y} &= \lambda_{2}\vec{x}^{\, t}\vec{y}\\
\end{align*}

Let's subtract the above two equations. We get,

\begin{align*}
	(\lambda_{2} - \lambda_1) \cdot \vec{x}^{\, t} \vec{y} &= 0 \\
	\vec{x}^{\, t} \vec{y} &= 0
\end{align*}


Hence since $\vec{x}^{\, t} \vec{y} = 0$, $x$ and $y$ are orthogonal.
\end{document}